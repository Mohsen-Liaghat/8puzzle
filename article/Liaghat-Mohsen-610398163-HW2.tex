\documentclass{article}

\usepackage{amsmath}
\usepackage{amssymb}
\usepackage{hyperref}
\usepackage{xcolor}
\usepackage{xepersian}

\settextfont{Arabic Typesetting}

\title{استفاده از الگوریتم های جستوجوی ناآگاهانه برای حل مساله 8-puzzle}
\author{محسن لیاقت 610398163}

\begin{document}
	\maketitle
    \tableofcontents
    
    \section{تعاریف}
        \begin{description}
        	\item[states : ] تمام حالات ممکن قابل دسترس از حالت آغازین 
        	( $\frac{9!}{2}$ حالت )
        	\item[Actions : ] بالا، پایین، چپ یا راست  بردن کاشی 0
        	\item[ test Goal :] همه کاشی ها در محل درست باشند
        	\item[cost path :] -1
        	\item[model transition :] در قالب یک تابع با همین عنوان در کلاس \lr{ENV\_8puzzle} در فایل \lr{puzzle8.py} آمده است.
        \end{description}
    \section{انتخاب تابع Heuristic}
        تابع Heuristic مورد نظر من مجموع کوتاه ترین فاصله هر کاشی با محل درست آن است که با نام hf در کلاس \lr{ENV\_8puzzle} در فایل \lr{puzzle8.py} آمده است.
        دلیل انتخاب این تابع موارد زیر است : 
        \begin{itemize}
        	\item به وضوح admissible است زیرا ما مجموع کوتاه ترین فاصله ها را حساب می کنیم و هزینه واقعی از این کمتر نخواهد بود. 
        	\item consistent است زیرا باتوجه به اینکه هزینه هر عمل 1 است خواهیم داشت.
        	\[ 
        	\textrm{cost}_{( n , a , n' )} + \textrm{hf}_{( n' )} = 1 + \textrm{hf}_{( n' )} 
        	\begin{cases}
        		\textrm{if a tile become closer to its place } & = 1 + \textrm{hf}_{( n )} - 1  = \textrm{hf}_{( n )} \\
        		\textrm{O.W} & \geqslant 1 + \textrm{hf}_{( n )} \geqslant \textrm{hf}_{( n )}
        	\end{cases}
        	\]
        	\item مقدار آن به هزینه واقعی تقریبا نزدیک است.
        \end{itemize} 
    \section{مثال}
        همه مثال های داده شده اجرا شده اند و نتیجه اجرای آنها به همراه حافظه مصرفی آنها در هر الگوریتم در فایل اکسل به پیوست آمده است.( به دلیل محدودیت های سخت افزار لپتاپ خودم صرفا توانستم حداکثر عمق recursion را 2500 قرار دهم.)\\
        به دلیل اینکه تعداد مثال ها زیاد بود از روی فایل مثال ها فایل های دیگری ساختم به نام 
        \lr{easyex.text , medex.text , hardex.text} که شامل همان مثال ها بود فقط فایل منظم تر شده بود و سپس همه ورودی ها را از آن فایل گرفتم و خروجی ها را در فایل 
        \lr{res.csv} ذخیره کردم سپس از روی آن فایل یک فایل اکسل ساختم.
        همه این فایل ها اگر سامانه ارور ندهد پیوست می شود :)\\
        به دلیل اینکه پیچیدگی زمانی IDS نمایی است برای مثال های medium و hard دیگر این الگوریتم را اجرا نکردم ( راستش چند بار اجرا کردم ولی هر بار بعد از گذشت چند ساعت هنوز همه مثال ها حل نشده بود. )
        \subsection{تحلیل نتایج}
            \begin{itemize}
            	\item الگوریتم DFS در حالات بسیاری به دلیل پر شدن استک terminate می کند نه به دلیل اینکه پاسخ را پیدا کرده است.
            	\item الگوریتم BFS‌ نسبت به DFS از حافظه بیشتری استفاده می کند و چون این حافظه به صورت هیپ است و سیستم عامل قابلیت صفحه بندی حافظه را دارد این الگوریتم با موفقیت پاسخ را می یا بد
            	\item الگوریتم UCS چون از min-heap استفاده می کند حافظه و زمان بیشتری را برای یافتن پاسخ نسبت به BFS نیاز دارد.
            	\item اگر به زمان پاسخ گویی IDS توجه شود مشاهده می شود که رشد این زمان نسب به رشد طول پاسخ بهینه نمایی است.
            	\item الگوریتم $A^*$ کمترین زمان و حافظه را نسبت به دیگر الگوریتم ها برای یافتن پاسخ نیاز دارد.
            \end{itemize}
        
\end{document}